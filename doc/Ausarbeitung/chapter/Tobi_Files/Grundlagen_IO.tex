\section{Blockierende und nichtblockierende \acs{I/O}}
\label{sec:nonblocking}Ein nichtblockierendes I/O-Modell ermöglicht die Verarbeitung von Befehlen bei zeitintensiven Berechnungen oder Datenbankabfragen ohne dass der Main-Thread der Anwendung blockiert wird. Es muss nicht auf das Ergebnis der Berechnung gewartet werden, bevor der nächste Befehl ausgeführt werden kann. Der Mechanismus kann beispielsweise mit Hilfe einer Ereignisschleife umgesetzt werden, die die Vorgänge in sogenannten Worker-Threads auslagert und nach Beendigung der Aufgaben die Callback-Funktionen aufruft, die im Main-Thread abgearbeitet werden (siehe Abb. \ref{fig:nonblocking}). Diese Technik nennt sich \acf{CPS} und wird im Kapitel \ref{sec:jscps} am Beispiel von JavaScript erklärt. Ein nichtblockierendes I/O-Modell stellt die Grundlage für asynchrone Programmierung dar.
\begin{figure}[H]
\centering
\includegraphics[width=0.4\textwidth]{images/nonblocking.png}
\caption[Event-Loop]{Event-Loop}
\label{fig:nonblocking}
\end{figure}
\acresetall
