\section{Was bedeutet asynchrone Programmierung?}
Zu Beginn stellt sich die Frage nach der Erläuterung des Begriffs \textit{asynchrone Programmierung}. Betrachten wir eine imperative Programmiersprache ohne Multithreading, so lässt sich dieser Begriff leicht erklären. Eine Anwendung in einem imperativen Kontext hat ihren Entry-Point in der ersten Zeile des Programmcodes. Der Code wird streng sequentiell abgearbeitet. Jeder Befehl blockiert die Anwendung, solange bis seine Abarbeitung beendet ist. Dieses Verhalten einer Anwendung wird als \textit{synchrone} Abarbeitung einer Befehlsfolge bezeichnet. Der Nachteil ist hier, dass Wartezeiten zwischen Berechnungen entstehen. Lange Berechnungen reduzieren die Performanz der Anwendung, da stets auf ihre Beendigung gewartet werden muss. Hier kommt \textit{asynchrone Programmierung} zum Tragen. Der Begriff beschreibt eine sequentielle Befehlsabarbeitung, bei der keine Wartezeit zwischen Berechnungen entsteht. Grundlage hierfür stellen ein nichtblockierendes I/O-Modell, welches im nächsten Kapitel erklärt wird, sowie diverse Programmierparadigmen dar.
 \acresetall

