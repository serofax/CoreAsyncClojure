\section{Vergleich von \acs{CSP} und dem Actor model}
\comment{Falsches Kapitel. Ich würde es zu den Grundlagen tun hinter CSP}
Das Aktorenmodell und \ac{CSP} definieren beide Modelle mit denen nebenläufige und parallele Ausführung modelliert werden kann. Dadurch haben sie automatisch schon viele Gemeinsamkeiten. Aufgrund der unterschiedlichen Herangehensweise und Definition unterscheiden sie sich in einigen Bereichen stark. 

Als erstes wird zusammengefasst das Aktorenmodell im nachfolgenden Unterkapitel erläutern. Im anschließenden Unterkapitel werden die Unterschiede und Gemeinsamkeiten der beiden Modelle aufgezeigt.

\subsection{Das Actor Model}
\comment{Sowas doch lieber in die Grundlagen?}
Das Actor Model (Aktorenmodell) ist ein von \citeauthor{ACTORS} beschriebenes Modell zur Modellierung von künstlichen Intelligenzen, das sich optimal für nebenläufige und parallele Programmierung eignet. Veröffentlicht wurde das Modell 1973 in dem Paper\footcite{ACTORS} \textit{A Universal Modular ACTOR Formalism for Artificial Intelligence}. Von \citeauthor{ACTORSNEW} wurde das Modell mit dem Paper \textit{Actor Model of Computation: Scalable Robust Information Systems}\footcite{ACTORSNEW} 2011 aufbereitet. 

Aktoren bestehen aus folgenden fundamentalen Teilen:

\begin{description}
\item[Prozess]\hfill \\
Ein Aktor benötigt einen Prozess, der die Aufgaben des Aktors ausführt. Üblicherweise ist das ein Thread.
\item[Speicher]\hfill \\
Zustände des Aktors werden in einem Speicher gesichert, damit unterschiedlich auf Nachrichten reagiert werden kann.
\item[Kommunikation] \hfill \\
Aktoren müssen untereinander kommunizieren können und sich Nachrichten schicken.
\end{description}

Ein Aktor ist für sich alleine genommen kein Aktoren, erst in einem Aktorensystem existiert er. Das Aktorensystem stellt ein Kommunikationsmedium bereit, das nach \glqq best effort\grqq\ versucht die Nachrichten zu übertragen. Deswegen können Nachrichten verloren gehen oder andere Nachrichten überholen. Aktoren verfügen über eine Adresse, worüber andere Aktoren ihnen Nachrichten schicken können. Nachrichten werden in das Aktorsystem gelegt und ein Aktor bekommt irgendwie die Nachricht zugestellt, wenn er sie benötigt. Puffer, Mailboxen und Queues werden im Aktorenmodell \textbf{nicht} explizit definiert\footcite[Seite 3, Rechte Spalte]{ACTORSNEW}. Die technische Umsetzung übernehmen die Implementierungen.

Ein Aktor verarbeitet mit seinem Prozess nur eine Nachricht zu einem Zeitpunkt und weil Aktoren keinen gemeinsamen Speicher benutzen dürfen, können keine Race-Conditions auftreten.
Das Aktorenmodell ist nicht deterministisch, weil die Kommunikation zwischen Aktoren nicht genau definiert ist und die Nachrichten zufällig beim Empfänger eintreffen können, wodurch Determinismus nicht garantiert werden kann.

\subsection{Unterschiede und Gemeinsamkeiten}
Es sind beides theoretische Modelle zur Beschreibung von Parallelität und Nebenläufigkeit. Unterscheiden tun sie sich bereits in ihren formalen Beschreibungen.\comment{Macht das Sinn?}.
\ac{CSP} ist eine Prozess-Algebra und basiert auf mathematischen Gesetzen, wodurch keine Unklarheiten existieren. Hingegen das Aktorenmodell orientiert sich an der Physik und kann deswegen einige Bereiche nicht genau spezifizieren, sodass Axiome als Begründung eingebracht werden. Dadurch wird das Modell sehr einfach und leicht verständlich. Das Aktorenmodell ist aufgrund der Axiome nicht deterministisch. Determinismus muss zusätzlich spezifiziert werden. Abläufe in \ac{CSP} sind deterministisch. Um die Automaten einfach zu halten, kann Nichtdeterminismus mit einem besonderen Operator ($\sqcap$) speziell modelliert werden.

Das Aktorenmodell definiert nicht, wie eine Nachricht empfangen wird. Das kann über einen blockierenden Aufruf passieren oder eine eingehende Nachricht kann eine Methode aufrufen, wie es zum Beispiel in dem Aktorenframework Akka umgesetzt wird. In \ac{CSP} kann es nur blockierende bzw. parkende Operationen geben, weil ein Prozess auf mehrere Channels gleichzeitig warten kann.

Der Datenaustausch zwischen zwei Prozessen in \ac{CSP} ist simultan. Sodass ein Prozess blockiert, wenn kein anderer Prozess mit der gegensätzlichen Aktion auf den Channel zugreift. Ein Buffer muss mit einem weiteren Prozess modelliert werden. Die Kommunikation mit Nachrichten zwischen zwei Aktoren passiert asymmetrisch, weil das Aktorensystem die Zustellung der Nachrichten übernimmt. Eine mengenmäßige Begrenzung ist nicht vorgesehen. In technischen Implementierungen des Aktorenmodells wie Akka existieren selbstverständlich Begrenzungen seitens der Konfiguration oder der Hardware. Zudem sichert Akka auch zu das Nachrichten nicht verloren gehen können und sich nicht überholen können.

\comment{schau mal hier rein \url{http://en.wikipedia.org/wiki/Actor_model_and_process_calculi}}



Actor Reaktiv - CSP Blockierend
Message /Event driven
physikalisches Modell - mathematisches Modell

\comment{Fehlt ein Vergleich zwischen CSP und CPS?}

\subsection{Synchronisation mit wechselseitigem Ausschluss}
Blockierend (Pessimistische Strategie) - CSP Blockierend
Lock
Semaphore
Monitor

\subsection{Synchronisation mit einem \acs{STM}}
Redo Optimistische Strategie - CSP Blockierend
