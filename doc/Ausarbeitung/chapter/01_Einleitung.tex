\chapter{Einleitung}
Diese Ausarbeitung befasst sich mit \textit{core.async}, einer Bibliothek zur asynchronen Programmierung in der Programmiersprache Clojure und dessen Skriptsprache ClojureScript. Es werden zunächst die grundlegenden Begrifflichkeiten erklärt, andere State-of-the-Art Programmierparadigmen, Entwurfsmuster und Frameworks betrachtet und anschließend mit \textit{core.async} verglichen. Anhand von Beispielen aus dem \acf{API} des Frameworks wird der aktuelle Entwicklungsstand des noch im Alpha-Stadium befindlichen Frameworks gezeigt und dessen Potential bewertet.
\comment { Parallelität }

\acresetall