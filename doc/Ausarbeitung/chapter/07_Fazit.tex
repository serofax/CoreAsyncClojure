\chapter{Zusammenfassung und Fazit}
Die Bibliothek \textit{core.async} implementiert das Channel Prinzip aus Tony Hoare's Prozess-Algebra \acf{CSP} zur Modellierung nebenläufiger Prozesse. Es wurde gezeigt, dass auch wenn ein System über Multi-Core-Prozessoren verfügt, keine Rücksicht auf Race-Conditions bei der Modellierung genommen werden muss, dennoch aber Deadlocks auftreten können. Anschließend wurde ein Vergleich zwischen \acs{CSP} und dem Aktorenmodell angestellt. Das Aktorenmodell ist wie \acs{CSP} ein Prinzip zur Modellierung von Nebenläufigkeit. Keines der Modelle kann gegenüber dem Anderen präferiert werden. Der Entwickler muss nach den bestehenden Anforderungen entscheiden, welches Modell den fachlichen Anforderungen besser entspricht, so auch beim Einsatz von \textit{core.async}. Jedoch sollten auch die Unterschiede der Implementierungen in Betracht gezogen werden, die eventuelle Nachteile des Modells aufwiegen.\\
Zunächst wurde der Begriff Asynchrone Programmierung anhand von JavaScript-Code erklärt und demonstriert und es wurde \acf{CPS} vorgestellt, anschließend wurde die Basis-Funktionalität der \acs{API} von \textit{core.async} mit Code-Beispielen demonstriert. Die Bibliothek wirkte sehr stabil. Der Umfang der \acs{API} ist für eine Version im Alpha-Stadium beachtlich und kann bereits jetzt produktiv eingesetzt werden. Es ist hervorzuheben, dass - ausgenommen Thread-basierter Konstrukte - \textit{core.async} kompatibel zu ClojureScript ist und der Code demnach zu JavaScript kompiliert werden kann. Dies erleichtert die Modellierung von Quasi-Parallelen Prozessen (single threaded) in JavaScript erheblich und schafft neue Möglichkeiten.\\
Dieses Projekt hat uns tiefe Einblicke in diverse hochinteressante Programmierparadigmen zur Modellierung von Nebenläufigkeit und Parallelität gewährt und uns persönlich weiter gebracht.

