\documentclass[a4paper,12pt,numbers=noenddot, chapteratlists=0pt]{scrreprt}
\usepackage[utf8]{inputenc}
\usepackage[T1]{fontenc}
\usepackage[ngerman]{babel}
\usepackage{csquotes}
\usepackage[printonlyused]{acronym}
\usepackage{hyperref}
\usepackage{footnote}
\usepackage[autocite=footnote, style=authortitle-icomp, citestyle=authortitle-icomp, maxcitenames=1]{biblatex}
\usepackage{here}

\usepackage{graphicx}
\usepackage{pdfpages}

\usepackage{chngcntr}
\counterwithout{figure}{chapter}
\counterwithout{table}{chapter}

\interfootnotelinepenalty=10000 
\bibliography{bibliography}
\begin{document}
\begin{titlepage}
\begin{center}
\Huge
\textbf{Asynchrone Programmierung mit core.async}\\
\vspace{1cm}
\Large
\vspace{1cm}
Programmieren in Clojure\\
\vspace{2cm}
Master of Science Informatik\\
Wintersemester 2013/2014\\
\vspace{2cm}
\textbf{Vincent Elliott Wagner}\\
\textbf{Tobias Schwalm}\\
\vspace{2cm}
04.04.2014
\end{center}
\vfill
Referent: Prof. Dr. Burkhard Renz\\\\
\end{titlepage}
\pagenumbering{Roman}
\clearpage
\addcontentsline{toc}{chapter}{Inhaltsverzeichnis}
\tableofcontents
\renewcommand{\thechapter}{\arabic{chapter}}
\pagenumbering{arabic}
\chapter{Einleitung}
\chapter{Grundlagen}
\section{Blockierendes und nicht-blockierendes I/O-Modell}
\section{Asynchrone Programmierung im Vergleich}
\subsection*{Paradigmen}
\section{Core.Async}
\subsection*{Vergleich Google Go}
\subsection*{Blockieren und Parken}
\subsection*{Channels, Puffer und Filter}
\subsection*{Go-Blocks}
\subsection*{Vergleich zu anderen Paradigmen}
\chapter{Beispiele}
\chapter{Fazit}
\end{document}
