%after adding or removing a entry call makeglossaries
%oder
%makeindex -s Bachelorarbeit.ist -t Bachelorarbeit.glg -o Bachelorarbeit.gls Bachelorarbeit.glo

\newglossaryentry{Binding}{name={Binding},
description={Die Schnittstelle eines Protokolls, worüber man ein System ansprechen kann}}

\newglossaryentry{Context-Pfad}{name={Context-Pfad},
description={In einer \acs{JEE}-Umgebung ist der Context-Pfad der relative Pfad der Webanwendung zum  Wurzelverzeichnis des Web-Servers}}

\newglossaryentry{cURL}{name={cURL},
description={Ein Kommandozeilenprogramm, um Dateien von/zu einem Server, über die \acs{URL}, zu empfangen/senden}}

\newglossaryentry{CRUD}{name={CRUD},
description={Create Read Update Delete, Beschreibt die grundlegende Funktion Daten oder Objekte zu erzeugen, zu lesen, zu aktualisieren und zu löschen}}


\newglossaryentry{Map}{name={Map},
description={Ein Datenobjekt, womit Objekte anhand eines Schlüssels verwaltet werden können}}

\newglossaryentry{Mapping}{name={Mapping},
description={Deutsch: Abbildung, Übertragen eines Systems auf ein anderes System}}

\newglossaryentry{Major-Version}{name={Major-Version},
description={Deutsch: Haupt-Version, Eine Version die signifikante Änderungen beinhaltet}}

\newglossaryentry{Minor-Version}{name={Minor-Version},
description={Deutsch: Neben-Version, Eine Version in der nur unbedeutende Änderungen gemacht wurden}}

\newglossaryentry{Overhead}{name={Overhead},
description={Deutsch: Verwaltungsdaten}}

\newglossaryentry{Paging}{name={Paging},
description={Verfahren bei dem eine Menge von Objekte, in verschiedenen Seiten (Pages) eingeteilt wird. Zusätzlich wird die Möglichkeit bereitstellt, nur bestimmte Seiten anzufordern. Dadurch müssen nicht alle Objekte verarbeitet werden, sondern die Menge kann eingegrenzt werden}}

\newglossaryentry{Parser}{name={Parser},
description={Programm, dass eine Information in ein verwendbares Format umwandelt}}

\newglossaryentry{Race Condition}{name={Race Condition},
description={Deutsch: Wettlaufsituation, Der gleichzeitige Zugriff auf ein Objekt durch mehrere Systeme}}


\newglossaryentry{Scheduler}{name={Scheduler},
description={Ein Programm, mit dem ein Vorgang zu einen definierten Zeitpunkt ausführt werden kann}}

\newglossaryentry{Scheduler-Job}{name={Scheduler-Job},
description={Ein Vorgang, der durch einen \gls{Scheduler} ausgeführt wird}}

\newglossaryentry{Servlet-Pfad}{name={Servlet-Pfad},
description={In einer \acs{JEE}-Umgebung ist der Servlet-Pfad der relative Pfad des Servlets zum Wurzelverzeichnis der Webanwendung}}

\newglossaryentry{Utilityklasse}{name={Utilityklasse},
description={Eine Klasse, die häufig verwendete Abläufe beinhaltet}}


